
    




    
\documentclass[11pt]{article}

    
    \usepackage[breakable]{tcolorbox}
    \tcbset{nobeforeafter} % prevents tcolorboxes being placing in paragraphs
    \usepackage{float}
    \floatplacement{figure}{H} % forces figures to be placed at the correct location
    
    \usepackage[T1]{fontenc}
    % Nicer default font (+ math font) than Computer Modern for most use cases
    \usepackage{mathpazo}

    % Basic figure setup, for now with no caption control since it's done
    % automatically by Pandoc (which extracts ![](path) syntax from Markdown).
    \usepackage{graphicx}
    % We will generate all images so they have a width \maxwidth. This means
    % that they will get their normal width if they fit onto the page, but
    % are scaled down if they would overflow the margins.
    \makeatletter
    \def\maxwidth{\ifdim\Gin@nat@width>\linewidth\linewidth
    \else\Gin@nat@width\fi}
    \makeatother
    \let\Oldincludegraphics\includegraphics
    % Set max figure width to be 80% of text width, for now hardcoded.
    \renewcommand{\includegraphics}[1]{\Oldincludegraphics[width=.8\maxwidth]{#1}}
    % Ensure that by default, figures have no caption (until we provide a
    % proper Figure object with a Caption API and a way to capture that
    % in the conversion process - todo).
    \usepackage{caption}
    \DeclareCaptionLabelFormat{nolabel}{}
    \captionsetup{labelformat=nolabel}

    \usepackage{adjustbox} % Used to constrain images to a maximum size 
    \usepackage{xcolor} % Allow colors to be defined
    \usepackage{enumerate} % Needed for markdown enumerations to work
    \usepackage{geometry} % Used to adjust the document margins
    \usepackage{amsmath} % Equations
    \usepackage{amssymb} % Equations
    \usepackage{textcomp} % defines textquotesingle
    % Hack from http://tex.stackexchange.com/a/47451/13684:
    \AtBeginDocument{%
        \def\PYZsq{\textquotesingle}% Upright quotes in Pygmentized code
    }
    \usepackage{upquote} % Upright quotes for verbatim code
    \usepackage{eurosym} % defines \euro
    \usepackage[mathletters]{ucs} % Extended unicode (utf-8) support
    \usepackage[utf8x]{inputenc} % Allow utf-8 characters in the tex document
    \usepackage{fancyvrb} % verbatim replacement that allows latex
    \usepackage{grffile} % extends the file name processing of package graphics 
                         % to support a larger range 
    % The hyperref package gives us a pdf with properly built
    % internal navigation ('pdf bookmarks' for the table of contents,
    % internal cross-reference links, web links for URLs, etc.)
    \usepackage{hyperref}
    \usepackage{longtable} % longtable support required by pandoc >1.10
    \usepackage{booktabs}  % table support for pandoc > 1.12.2
    \usepackage[inline]{enumitem} % IRkernel/repr support (it uses the enumerate* environment)
    \usepackage[normalem]{ulem} % ulem is needed to support strikethroughs (\sout)
                                % normalem makes italics be italics, not underlines
    \usepackage{mathrsfs}
    

    
    % Colors for the hyperref package
    \definecolor{urlcolor}{rgb}{0,.145,.698}
    \definecolor{linkcolor}{rgb}{.71,0.21,0.01}
    \definecolor{citecolor}{rgb}{.12,.54,.11}

    % ANSI colors
    \definecolor{ansi-black}{HTML}{3E424D}
    \definecolor{ansi-black-intense}{HTML}{282C36}
    \definecolor{ansi-red}{HTML}{E75C58}
    \definecolor{ansi-red-intense}{HTML}{B22B31}
    \definecolor{ansi-green}{HTML}{00A250}
    \definecolor{ansi-green-intense}{HTML}{007427}
    \definecolor{ansi-yellow}{HTML}{DDB62B}
    \definecolor{ansi-yellow-intense}{HTML}{B27D12}
    \definecolor{ansi-blue}{HTML}{208FFB}
    \definecolor{ansi-blue-intense}{HTML}{0065CA}
    \definecolor{ansi-magenta}{HTML}{D160C4}
    \definecolor{ansi-magenta-intense}{HTML}{A03196}
    \definecolor{ansi-cyan}{HTML}{60C6C8}
    \definecolor{ansi-cyan-intense}{HTML}{258F8F}
    \definecolor{ansi-white}{HTML}{C5C1B4}
    \definecolor{ansi-white-intense}{HTML}{A1A6B2}
    \definecolor{ansi-default-inverse-fg}{HTML}{FFFFFF}
    \definecolor{ansi-default-inverse-bg}{HTML}{000000}

    % commands and environments needed by pandoc snippets
    % extracted from the output of `pandoc -s`
    \providecommand{\tightlist}{%
      \setlength{\itemsep}{0pt}\setlength{\parskip}{0pt}}
    \DefineVerbatimEnvironment{Highlighting}{Verbatim}{commandchars=\\\{\}}
    % Add ',fontsize=\small' for more characters per line
    \newenvironment{Shaded}{}{}
    \newcommand{\KeywordTok}[1]{\textcolor[rgb]{0.00,0.44,0.13}{\textbf{{#1}}}}
    \newcommand{\DataTypeTok}[1]{\textcolor[rgb]{0.56,0.13,0.00}{{#1}}}
    \newcommand{\DecValTok}[1]{\textcolor[rgb]{0.25,0.63,0.44}{{#1}}}
    \newcommand{\BaseNTok}[1]{\textcolor[rgb]{0.25,0.63,0.44}{{#1}}}
    \newcommand{\FloatTok}[1]{\textcolor[rgb]{0.25,0.63,0.44}{{#1}}}
    \newcommand{\CharTok}[1]{\textcolor[rgb]{0.25,0.44,0.63}{{#1}}}
    \newcommand{\StringTok}[1]{\textcolor[rgb]{0.25,0.44,0.63}{{#1}}}
    \newcommand{\CommentTok}[1]{\textcolor[rgb]{0.38,0.63,0.69}{\textit{{#1}}}}
    \newcommand{\OtherTok}[1]{\textcolor[rgb]{0.00,0.44,0.13}{{#1}}}
    \newcommand{\AlertTok}[1]{\textcolor[rgb]{1.00,0.00,0.00}{\textbf{{#1}}}}
    \newcommand{\FunctionTok}[1]{\textcolor[rgb]{0.02,0.16,0.49}{{#1}}}
    \newcommand{\RegionMarkerTok}[1]{{#1}}
    \newcommand{\ErrorTok}[1]{\textcolor[rgb]{1.00,0.00,0.00}{\textbf{{#1}}}}
    \newcommand{\NormalTok}[1]{{#1}}
    
    % Additional commands for more recent versions of Pandoc
    \newcommand{\ConstantTok}[1]{\textcolor[rgb]{0.53,0.00,0.00}{{#1}}}
    \newcommand{\SpecialCharTok}[1]{\textcolor[rgb]{0.25,0.44,0.63}{{#1}}}
    \newcommand{\VerbatimStringTok}[1]{\textcolor[rgb]{0.25,0.44,0.63}{{#1}}}
    \newcommand{\SpecialStringTok}[1]{\textcolor[rgb]{0.73,0.40,0.53}{{#1}}}
    \newcommand{\ImportTok}[1]{{#1}}
    \newcommand{\DocumentationTok}[1]{\textcolor[rgb]{0.73,0.13,0.13}{\textit{{#1}}}}
    \newcommand{\AnnotationTok}[1]{\textcolor[rgb]{0.38,0.63,0.69}{\textbf{\textit{{#1}}}}}
    \newcommand{\CommentVarTok}[1]{\textcolor[rgb]{0.38,0.63,0.69}{\textbf{\textit{{#1}}}}}
    \newcommand{\VariableTok}[1]{\textcolor[rgb]{0.10,0.09,0.49}{{#1}}}
    \newcommand{\ControlFlowTok}[1]{\textcolor[rgb]{0.00,0.44,0.13}{\textbf{{#1}}}}
    \newcommand{\OperatorTok}[1]{\textcolor[rgb]{0.40,0.40,0.40}{{#1}}}
    \newcommand{\BuiltInTok}[1]{{#1}}
    \newcommand{\ExtensionTok}[1]{{#1}}
    \newcommand{\PreprocessorTok}[1]{\textcolor[rgb]{0.74,0.48,0.00}{{#1}}}
    \newcommand{\AttributeTok}[1]{\textcolor[rgb]{0.49,0.56,0.16}{{#1}}}
    \newcommand{\InformationTok}[1]{\textcolor[rgb]{0.38,0.63,0.69}{\textbf{\textit{{#1}}}}}
    \newcommand{\WarningTok}[1]{\textcolor[rgb]{0.38,0.63,0.69}{\textbf{\textit{{#1}}}}}
    
    
    % Define a nice break command that doesn't care if a line doesn't already
    % exist.
    \def\br{\hspace*{\fill} \\* }
    % Math Jax compatibility definitions
    \def\gt{>}
    \def\lt{<}
    \let\Oldtex\TeX
    \let\Oldlatex\LaTeX
    \renewcommand{\TeX}{\textrm{\Oldtex}}
    \renewcommand{\LaTeX}{\textrm{\Oldlatex}}
    % Document parameters
    % Document title
    \title{theory\_ru}
    
    
    
    
    
% Pygments definitions
\makeatletter
\def\PY@reset{\let\PY@it=\relax \let\PY@bf=\relax%
    \let\PY@ul=\relax \let\PY@tc=\relax%
    \let\PY@bc=\relax \let\PY@ff=\relax}
\def\PY@tok#1{\csname PY@tok@#1\endcsname}
\def\PY@toks#1+{\ifx\relax#1\empty\else%
    \PY@tok{#1}\expandafter\PY@toks\fi}
\def\PY@do#1{\PY@bc{\PY@tc{\PY@ul{%
    \PY@it{\PY@bf{\PY@ff{#1}}}}}}}
\def\PY#1#2{\PY@reset\PY@toks#1+\relax+\PY@do{#2}}

\expandafter\def\csname PY@tok@w\endcsname{\def\PY@tc##1{\textcolor[rgb]{0.73,0.73,0.73}{##1}}}
\expandafter\def\csname PY@tok@c\endcsname{\let\PY@it=\textit\def\PY@tc##1{\textcolor[rgb]{0.25,0.50,0.50}{##1}}}
\expandafter\def\csname PY@tok@cp\endcsname{\def\PY@tc##1{\textcolor[rgb]{0.74,0.48,0.00}{##1}}}
\expandafter\def\csname PY@tok@k\endcsname{\let\PY@bf=\textbf\def\PY@tc##1{\textcolor[rgb]{0.00,0.50,0.00}{##1}}}
\expandafter\def\csname PY@tok@kp\endcsname{\def\PY@tc##1{\textcolor[rgb]{0.00,0.50,0.00}{##1}}}
\expandafter\def\csname PY@tok@kt\endcsname{\def\PY@tc##1{\textcolor[rgb]{0.69,0.00,0.25}{##1}}}
\expandafter\def\csname PY@tok@o\endcsname{\def\PY@tc##1{\textcolor[rgb]{0.40,0.40,0.40}{##1}}}
\expandafter\def\csname PY@tok@ow\endcsname{\let\PY@bf=\textbf\def\PY@tc##1{\textcolor[rgb]{0.67,0.13,1.00}{##1}}}
\expandafter\def\csname PY@tok@nb\endcsname{\def\PY@tc##1{\textcolor[rgb]{0.00,0.50,0.00}{##1}}}
\expandafter\def\csname PY@tok@nf\endcsname{\def\PY@tc##1{\textcolor[rgb]{0.00,0.00,1.00}{##1}}}
\expandafter\def\csname PY@tok@nc\endcsname{\let\PY@bf=\textbf\def\PY@tc##1{\textcolor[rgb]{0.00,0.00,1.00}{##1}}}
\expandafter\def\csname PY@tok@nn\endcsname{\let\PY@bf=\textbf\def\PY@tc##1{\textcolor[rgb]{0.00,0.00,1.00}{##1}}}
\expandafter\def\csname PY@tok@ne\endcsname{\let\PY@bf=\textbf\def\PY@tc##1{\textcolor[rgb]{0.82,0.25,0.23}{##1}}}
\expandafter\def\csname PY@tok@nv\endcsname{\def\PY@tc##1{\textcolor[rgb]{0.10,0.09,0.49}{##1}}}
\expandafter\def\csname PY@tok@no\endcsname{\def\PY@tc##1{\textcolor[rgb]{0.53,0.00,0.00}{##1}}}
\expandafter\def\csname PY@tok@nl\endcsname{\def\PY@tc##1{\textcolor[rgb]{0.63,0.63,0.00}{##1}}}
\expandafter\def\csname PY@tok@ni\endcsname{\let\PY@bf=\textbf\def\PY@tc##1{\textcolor[rgb]{0.60,0.60,0.60}{##1}}}
\expandafter\def\csname PY@tok@na\endcsname{\def\PY@tc##1{\textcolor[rgb]{0.49,0.56,0.16}{##1}}}
\expandafter\def\csname PY@tok@nt\endcsname{\let\PY@bf=\textbf\def\PY@tc##1{\textcolor[rgb]{0.00,0.50,0.00}{##1}}}
\expandafter\def\csname PY@tok@nd\endcsname{\def\PY@tc##1{\textcolor[rgb]{0.67,0.13,1.00}{##1}}}
\expandafter\def\csname PY@tok@s\endcsname{\def\PY@tc##1{\textcolor[rgb]{0.73,0.13,0.13}{##1}}}
\expandafter\def\csname PY@tok@sd\endcsname{\let\PY@it=\textit\def\PY@tc##1{\textcolor[rgb]{0.73,0.13,0.13}{##1}}}
\expandafter\def\csname PY@tok@si\endcsname{\let\PY@bf=\textbf\def\PY@tc##1{\textcolor[rgb]{0.73,0.40,0.53}{##1}}}
\expandafter\def\csname PY@tok@se\endcsname{\let\PY@bf=\textbf\def\PY@tc##1{\textcolor[rgb]{0.73,0.40,0.13}{##1}}}
\expandafter\def\csname PY@tok@sr\endcsname{\def\PY@tc##1{\textcolor[rgb]{0.73,0.40,0.53}{##1}}}
\expandafter\def\csname PY@tok@ss\endcsname{\def\PY@tc##1{\textcolor[rgb]{0.10,0.09,0.49}{##1}}}
\expandafter\def\csname PY@tok@sx\endcsname{\def\PY@tc##1{\textcolor[rgb]{0.00,0.50,0.00}{##1}}}
\expandafter\def\csname PY@tok@m\endcsname{\def\PY@tc##1{\textcolor[rgb]{0.40,0.40,0.40}{##1}}}
\expandafter\def\csname PY@tok@gh\endcsname{\let\PY@bf=\textbf\def\PY@tc##1{\textcolor[rgb]{0.00,0.00,0.50}{##1}}}
\expandafter\def\csname PY@tok@gu\endcsname{\let\PY@bf=\textbf\def\PY@tc##1{\textcolor[rgb]{0.50,0.00,0.50}{##1}}}
\expandafter\def\csname PY@tok@gd\endcsname{\def\PY@tc##1{\textcolor[rgb]{0.63,0.00,0.00}{##1}}}
\expandafter\def\csname PY@tok@gi\endcsname{\def\PY@tc##1{\textcolor[rgb]{0.00,0.63,0.00}{##1}}}
\expandafter\def\csname PY@tok@gr\endcsname{\def\PY@tc##1{\textcolor[rgb]{1.00,0.00,0.00}{##1}}}
\expandafter\def\csname PY@tok@ge\endcsname{\let\PY@it=\textit}
\expandafter\def\csname PY@tok@gs\endcsname{\let\PY@bf=\textbf}
\expandafter\def\csname PY@tok@gp\endcsname{\let\PY@bf=\textbf\def\PY@tc##1{\textcolor[rgb]{0.00,0.00,0.50}{##1}}}
\expandafter\def\csname PY@tok@go\endcsname{\def\PY@tc##1{\textcolor[rgb]{0.53,0.53,0.53}{##1}}}
\expandafter\def\csname PY@tok@gt\endcsname{\def\PY@tc##1{\textcolor[rgb]{0.00,0.27,0.87}{##1}}}
\expandafter\def\csname PY@tok@err\endcsname{\def\PY@bc##1{\setlength{\fboxsep}{0pt}\fcolorbox[rgb]{1.00,0.00,0.00}{1,1,1}{\strut ##1}}}
\expandafter\def\csname PY@tok@kc\endcsname{\let\PY@bf=\textbf\def\PY@tc##1{\textcolor[rgb]{0.00,0.50,0.00}{##1}}}
\expandafter\def\csname PY@tok@kd\endcsname{\let\PY@bf=\textbf\def\PY@tc##1{\textcolor[rgb]{0.00,0.50,0.00}{##1}}}
\expandafter\def\csname PY@tok@kn\endcsname{\let\PY@bf=\textbf\def\PY@tc##1{\textcolor[rgb]{0.00,0.50,0.00}{##1}}}
\expandafter\def\csname PY@tok@kr\endcsname{\let\PY@bf=\textbf\def\PY@tc##1{\textcolor[rgb]{0.00,0.50,0.00}{##1}}}
\expandafter\def\csname PY@tok@bp\endcsname{\def\PY@tc##1{\textcolor[rgb]{0.00,0.50,0.00}{##1}}}
\expandafter\def\csname PY@tok@fm\endcsname{\def\PY@tc##1{\textcolor[rgb]{0.00,0.00,1.00}{##1}}}
\expandafter\def\csname PY@tok@vc\endcsname{\def\PY@tc##1{\textcolor[rgb]{0.10,0.09,0.49}{##1}}}
\expandafter\def\csname PY@tok@vg\endcsname{\def\PY@tc##1{\textcolor[rgb]{0.10,0.09,0.49}{##1}}}
\expandafter\def\csname PY@tok@vi\endcsname{\def\PY@tc##1{\textcolor[rgb]{0.10,0.09,0.49}{##1}}}
\expandafter\def\csname PY@tok@vm\endcsname{\def\PY@tc##1{\textcolor[rgb]{0.10,0.09,0.49}{##1}}}
\expandafter\def\csname PY@tok@sa\endcsname{\def\PY@tc##1{\textcolor[rgb]{0.73,0.13,0.13}{##1}}}
\expandafter\def\csname PY@tok@sb\endcsname{\def\PY@tc##1{\textcolor[rgb]{0.73,0.13,0.13}{##1}}}
\expandafter\def\csname PY@tok@sc\endcsname{\def\PY@tc##1{\textcolor[rgb]{0.73,0.13,0.13}{##1}}}
\expandafter\def\csname PY@tok@dl\endcsname{\def\PY@tc##1{\textcolor[rgb]{0.73,0.13,0.13}{##1}}}
\expandafter\def\csname PY@tok@s2\endcsname{\def\PY@tc##1{\textcolor[rgb]{0.73,0.13,0.13}{##1}}}
\expandafter\def\csname PY@tok@sh\endcsname{\def\PY@tc##1{\textcolor[rgb]{0.73,0.13,0.13}{##1}}}
\expandafter\def\csname PY@tok@s1\endcsname{\def\PY@tc##1{\textcolor[rgb]{0.73,0.13,0.13}{##1}}}
\expandafter\def\csname PY@tok@mb\endcsname{\def\PY@tc##1{\textcolor[rgb]{0.40,0.40,0.40}{##1}}}
\expandafter\def\csname PY@tok@mf\endcsname{\def\PY@tc##1{\textcolor[rgb]{0.40,0.40,0.40}{##1}}}
\expandafter\def\csname PY@tok@mh\endcsname{\def\PY@tc##1{\textcolor[rgb]{0.40,0.40,0.40}{##1}}}
\expandafter\def\csname PY@tok@mi\endcsname{\def\PY@tc##1{\textcolor[rgb]{0.40,0.40,0.40}{##1}}}
\expandafter\def\csname PY@tok@il\endcsname{\def\PY@tc##1{\textcolor[rgb]{0.40,0.40,0.40}{##1}}}
\expandafter\def\csname PY@tok@mo\endcsname{\def\PY@tc##1{\textcolor[rgb]{0.40,0.40,0.40}{##1}}}
\expandafter\def\csname PY@tok@ch\endcsname{\let\PY@it=\textit\def\PY@tc##1{\textcolor[rgb]{0.25,0.50,0.50}{##1}}}
\expandafter\def\csname PY@tok@cm\endcsname{\let\PY@it=\textit\def\PY@tc##1{\textcolor[rgb]{0.25,0.50,0.50}{##1}}}
\expandafter\def\csname PY@tok@cpf\endcsname{\let\PY@it=\textit\def\PY@tc##1{\textcolor[rgb]{0.25,0.50,0.50}{##1}}}
\expandafter\def\csname PY@tok@c1\endcsname{\let\PY@it=\textit\def\PY@tc##1{\textcolor[rgb]{0.25,0.50,0.50}{##1}}}
\expandafter\def\csname PY@tok@cs\endcsname{\let\PY@it=\textit\def\PY@tc##1{\textcolor[rgb]{0.25,0.50,0.50}{##1}}}

\def\PYZbs{\char`\\}
\def\PYZus{\char`\_}
\def\PYZob{\char`\{}
\def\PYZcb{\char`\}}
\def\PYZca{\char`\^}
\def\PYZam{\char`\&}
\def\PYZlt{\char`\<}
\def\PYZgt{\char`\>}
\def\PYZsh{\char`\#}
\def\PYZpc{\char`\%}
\def\PYZdl{\char`\$}
\def\PYZhy{\char`\-}
\def\PYZsq{\char`\'}
\def\PYZdq{\char`\"}
\def\PYZti{\char`\~}
% for compatibility with earlier versions
\def\PYZat{@}
\def\PYZlb{[}
\def\PYZrb{]}
\makeatother


    % For linebreaks inside Verbatim environment from package fancyvrb. 
    \makeatletter
        \newbox\Wrappedcontinuationbox 
        \newbox\Wrappedvisiblespacebox 
        \newcommand*\Wrappedvisiblespace {\textcolor{red}{\textvisiblespace}} 
        \newcommand*\Wrappedcontinuationsymbol {\textcolor{red}{\llap{\tiny$\m@th\hookrightarrow$}}} 
        \newcommand*\Wrappedcontinuationindent {3ex } 
        \newcommand*\Wrappedafterbreak {\kern\Wrappedcontinuationindent\copy\Wrappedcontinuationbox} 
        % Take advantage of the already applied Pygments mark-up to insert 
        % potential linebreaks for TeX processing. 
        %        {, <, #, %, $, ' and ": go to next line. 
        %        _, }, ^, &, >, - and ~: stay at end of broken line. 
        % Use of \textquotesingle for straight quote. 
        \newcommand*\Wrappedbreaksatspecials {% 
            \def\PYGZus{\discretionary{\char`\_}{\Wrappedafterbreak}{\char`\_}}% 
            \def\PYGZob{\discretionary{}{\Wrappedafterbreak\char`\{}{\char`\{}}% 
            \def\PYGZcb{\discretionary{\char`\}}{\Wrappedafterbreak}{\char`\}}}% 
            \def\PYGZca{\discretionary{\char`\^}{\Wrappedafterbreak}{\char`\^}}% 
            \def\PYGZam{\discretionary{\char`\&}{\Wrappedafterbreak}{\char`\&}}% 
            \def\PYGZlt{\discretionary{}{\Wrappedafterbreak\char`\<}{\char`\<}}% 
            \def\PYGZgt{\discretionary{\char`\>}{\Wrappedafterbreak}{\char`\>}}% 
            \def\PYGZsh{\discretionary{}{\Wrappedafterbreak\char`\#}{\char`\#}}% 
            \def\PYGZpc{\discretionary{}{\Wrappedafterbreak\char`\%}{\char`\%}}% 
            \def\PYGZdl{\discretionary{}{\Wrappedafterbreak\char`\$}{\char`\$}}% 
            \def\PYGZhy{\discretionary{\char`\-}{\Wrappedafterbreak}{\char`\-}}% 
            \def\PYGZsq{\discretionary{}{\Wrappedafterbreak\textquotesingle}{\textquotesingle}}% 
            \def\PYGZdq{\discretionary{}{\Wrappedafterbreak\char`\"}{\char`\"}}% 
            \def\PYGZti{\discretionary{\char`\~}{\Wrappedafterbreak}{\char`\~}}% 
        } 
        % Some characters . , ; ? ! / are not pygmentized. 
        % This macro makes them "active" and they will insert potential linebreaks 
        \newcommand*\Wrappedbreaksatpunct {% 
            \lccode`\~`\.\lowercase{\def~}{\discretionary{\hbox{\char`\.}}{\Wrappedafterbreak}{\hbox{\char`\.}}}% 
            \lccode`\~`\,\lowercase{\def~}{\discretionary{\hbox{\char`\,}}{\Wrappedafterbreak}{\hbox{\char`\,}}}% 
            \lccode`\~`\;\lowercase{\def~}{\discretionary{\hbox{\char`\;}}{\Wrappedafterbreak}{\hbox{\char`\;}}}% 
            \lccode`\~`\:\lowercase{\def~}{\discretionary{\hbox{\char`\:}}{\Wrappedafterbreak}{\hbox{\char`\:}}}% 
            \lccode`\~`\?\lowercase{\def~}{\discretionary{\hbox{\char`\?}}{\Wrappedafterbreak}{\hbox{\char`\?}}}% 
            \lccode`\~`\!\lowercase{\def~}{\discretionary{\hbox{\char`\!}}{\Wrappedafterbreak}{\hbox{\char`\!}}}% 
            \lccode`\~`\/\lowercase{\def~}{\discretionary{\hbox{\char`\/}}{\Wrappedafterbreak}{\hbox{\char`\/}}}% 
            \catcode`\.\active
            \catcode`\,\active 
            \catcode`\;\active
            \catcode`\:\active
            \catcode`\?\active
            \catcode`\!\active
            \catcode`\/\active 
            \lccode`\~`\~ 	
        }
    \makeatother

    \let\OriginalVerbatim=\Verbatim
    \makeatletter
    \renewcommand{\Verbatim}[1][1]{%
        %\parskip\z@skip
        \sbox\Wrappedcontinuationbox {\Wrappedcontinuationsymbol}%
        \sbox\Wrappedvisiblespacebox {\FV@SetupFont\Wrappedvisiblespace}%
        \def\FancyVerbFormatLine ##1{\hsize\linewidth
            \vtop{\raggedright\hyphenpenalty\z@\exhyphenpenalty\z@
                \doublehyphendemerits\z@\finalhyphendemerits\z@
                \strut ##1\strut}%
        }%
        % If the linebreak is at a space, the latter will be displayed as visible
        % space at end of first line, and a continuation symbol starts next line.
        % Stretch/shrink are however usually zero for typewriter font.
        \def\FV@Space {%
            \nobreak\hskip\z@ plus\fontdimen3\font minus\fontdimen4\font
            \discretionary{\copy\Wrappedvisiblespacebox}{\Wrappedafterbreak}
            {\kern\fontdimen2\font}%
        }%
        
        % Allow breaks at special characters using \PYG... macros.
        \Wrappedbreaksatspecials
        % Breaks at punctuation characters . , ; ? ! and / need catcode=\active 	
        \OriginalVerbatim[#1,codes*=\Wrappedbreaksatpunct]%
    }
    \makeatother

    % Exact colors from NB
    \definecolor{incolor}{HTML}{303F9F}
    \definecolor{outcolor}{HTML}{D84315}
    \definecolor{cellborder}{HTML}{CFCFCF}
    \definecolor{cellbackground}{HTML}{F7F7F7}
    
    % prompt
    \newcommand{\prompt}[4]{
        \llap{{\color{#2}[#3]: #4}}\vspace{-1.25em}
    }
    

    
    % Prevent overflowing lines due to hard-to-break entities
    \sloppy 
    % Setup hyperref package
    \hypersetup{
      breaklinks=true,  % so long urls are correctly broken across lines
      colorlinks=true,
      urlcolor=urlcolor,
      linkcolor=linkcolor,
      citecolor=citecolor,
      }
    % Slightly bigger margins than the latex defaults
    
    \geometry{verbose,tmargin=1in,bmargin=1in,lmargin=1in,rmargin=1in}
    
    

    \begin{document}
    
    
    \maketitle
    
    

    
    \hypertarget{ux43eux433ux438ux431ux430ux44eux449ux430ux44f-ux43fux443ux447ux43aux430-ux441-ux443ux447ux435ux442ux43eux43c-ux432ux43bux438ux44fux43dux438ux44f-ux43fux440ux43eux441ux442ux440ux430ux43dux441ux442ux432ux435ux43dux43dux43eux433ux43e-ux437ux430ux440ux44fux434ux430}{%
\section{Огибающая пучка с учетом влияния пространственного
заряда}\label{ux43eux433ux438ux431ux430ux44eux449ux430ux44f-ux43fux443ux447ux43aux430-ux441-ux443ux447ux435ux442ux43eux43c-ux432ux43bux438ux44fux43dux438ux44f-ux43fux440ux43eux441ux442ux440ux430ux43dux441ux442ux432ux435ux43dux43dux43eux433ux43e-ux437ux430ux440ux44fux434ux430}}

V. Fedorov, D. Nikiforov, A. Petrenko, (Novosibirsk, 2019)

    Сначала разберем теорию, затем расчитаем огибающую электронного пучка в
ЛИУ-5 с помощью Python и Astra, приведем сравнение результатов

    \hypertarget{ux443ux440ux430ux432ux43dux435ux43dux438ux44f-ux43cux430ux43aux441ux432ux435ux43bux43bux430}{%
\subsubsection{Уравнения
Максвелла}\label{ux443ux440ux430ux432ux43dux435ux43dux438ux44f-ux43cux430ux43aux441ux432ux435ux43bux43bux430}}

    Объемная плотность тока пучка \(\jmath = \rho\upsilon\), где \(\rho\) -
объемная плотность заряда,\(\upsilon\) - скорость пучка. Запишем
дифференциальные уравнения Максвелла: \[
\nabla \vec{D} = 4\pi\rho,
\] \[
\nabla\times \vec{H} = \frac{4\pi\vec{\jmath}}{c},
\] где \(\vec{D}\) - индукция электрического поля, \(\vec{H}\) -
напряженность магнитного поля, \(с\) - скорость света. Используем
теорему Стокса об интегрировании дифференциальных форм, чтобы получить
уравнения Максвелла в интегральной форме: \[
\oint\limits_{\eth V} \vec{D}\vec{dS} = 4\pi\int\limits_V\rho{dV},
\] \[
\oint\limits_{\eth S} \vec{H}\vec{dl} = \frac{4\pi}{c}\int\limits_S\vec{\jmath}\vec{dS}.
\]

    Найдем \(D_r\) для цилиндрического пучка радиуса \(a\) с постоянной
плотностью \(\rho_0\): \[
D_r = \frac{4\pi}{r}\int\limits^r_0\rho(\xi)\xi d\xi = 
\begin{equation*}
 \begin{cases}
   \displaystyle 2\pi\rho_0 r, r < a, 
   \\
   \displaystyle \frac{2\pi\rho_0 a^2}{r}, r > a.
 \end{cases}
\end{equation*}
\]

    Учитывая, что в вакууме \(D = E\), \(H = B\), \(E\) - напряженность
электрического поля, \(B\) - индукция магнитного поля, и в плоском
пространстве в декартовой системе координат \(H_\alpha = \beta D_r\),
где \(\displaystyle\beta = \frac{\upsilon}{c}\), радиальная компонента
силы \(F_r\) из силы Лоренца
\(\vec{F} = e\vec{E} + \displaystyle\frac{e}{c}\vec{\upsilon}\times\vec{B}\)
: \[
F_r = eE_r - \frac{e\upsilon_z B_\alpha}{c} = eE_r(1-\displaystyle \frac{\upsilon^2}{c^2}) = \displaystyle \frac{eE_r}{\gamma^2}.
\] Полезно выразить поле через ток \$ I = \rho\_0\upsilon\pi a\^{}2 \$,
тогда: \[
E_r = 
\begin{equation*}
 \begin{cases}
   \displaystyle \frac{2 I r}{a^2\upsilon}, r < a, 
   \\
   \displaystyle \frac{2 I}{r\upsilon}, r > a.
 \end{cases}
\end{equation*}
\]

    \hypertarget{ux443ux440ux430ux432ux43dux435ux43dux438ux44f-ux434ux432ux438ux436ux435ux43dux438ux44f}{%
\subsubsection{Уравнения
движения}\label{ux443ux440ux430ux432ux43dux435ux43dux438ux44f-ux434ux432ux438ux436ux435ux43dux438ux44f}}

    Второй закон Ньютона \(\dot p_r = F_r\), используем параксиальное
приближение и считаем \(\gamma = const\): \[
\gamma m \ddot r = \displaystyle \frac{eE_r}{\gamma^2} = \displaystyle \frac{2 I e}{\gamma^2 a^2 \upsilon} r, 
\] получилось линейное уравнение, но так как все частицы движутся нужно
учесть, что \(a = a(t)\). Решение линейного уравнения можно представить
как линейное преобразование фазовой плоскости. Так как отрезок на
фазовой плоскости при невырожденном линейном преобразовании переходит в
отрезок, то его можно охарактеризовать одной точкой. Следовательно,
выберем крайнюю точку \(r = a\), которая будет характеризовать крайнюю
траекторию: \[
\gamma m \ddot r = \displaystyle \frac{2 I e}{\gamma^2 a \upsilon}. 
\] Перейдем к дифференцированию по \(z\), учтем, что
\(\displaystyle dt = \frac{dz}{v}\), тогда: \[
a'' = \displaystyle \frac{e}{a}\frac{2I}{m\gamma^3\upsilon^3}.
\] Введем характерный альфвеновский ток
\(I_a = \displaystyle \frac{mc^3}{e} \approx\) 17 kA, следовательно: \[
a'' = \displaystyle \frac{2I}{I_a (\beta\gamma)^3} \frac{1}{a}.
\] Учтем внешнюю фокусировку, предполагая суперпозицию полей (верно не
всегда, например, в нелинейных средах это не выполняется), получим: \[
a'' + k(z)a - \displaystyle \frac{2I}{I_a (\beta\gamma)^3} \frac{1}{a} = 0,
\] что напоминает уравнение огибающей:
\[ w'' + kw - \displaystyle \frac{1}{w^3} = 0 ,\] где \$ w
=\displaystyle  \sqrt \beta .\$

    \hypertarget{ux443ux440ux430ux432ux43dux435ux43dux438ux44f-ux43eux433ux438ux431ux430ux44eux449ux435ux439-ux434ux43bux44f-ux44dux43bux43bux438ux43fux442ux438ux447ux435ux441ux43aux43eux433ux43e-ux43fux443ux447ux43aux430-ux441-ux440ux430ux441ux43fux440ux435ux434ux435ux43bux435ux43dux438ux435ux43c-ux43aux430ux43fux447ux438ux43dux441ux43aux43eux433ux43e-ux432ux43bux430ux434ux438ux43cux438ux440ux441ux43aux43eux433ux43e-ux441-ux432ux43dux435ux448ux43dux435ux439-ux444ux43eux43aux443ux441ux438ux440ux43eux432ux43aux43eux439-ux43bux438ux43dux435ux439ux43dux44bux43cux438-ux43fux43eux43bux44fux43cux438}{%
\subsubsection{Уравнения огибающей для эллиптического пучка с
распределением Капчинского-Владимирского с внешней фокусировкой
линейными
полями}\label{ux443ux440ux430ux432ux43dux435ux43dux438ux44f-ux43eux433ux438ux431ux430ux44eux449ux435ux439-ux434ux43bux44f-ux44dux43bux43bux438ux43fux442ux438ux447ux435ux441ux43aux43eux433ux43e-ux43fux443ux447ux43aux430-ux441-ux440ux430ux441ux43fux440ux435ux434ux435ux43bux435ux43dux438ux435ux43c-ux43aux430ux43fux447ux438ux43dux441ux43aux43eux433ux43e-ux432ux43bux430ux434ux438ux43cux438ux440ux441ux43aux43eux433ux43e-ux441-ux432ux43dux435ux448ux43dux435ux439-ux444ux43eux43aux443ux441ux438ux440ux43eux432ux43aux43eux439-ux43bux438ux43dux435ux439ux43dux44bux43cux438-ux43fux43eux43bux44fux43cux438}}

    Распределение Капчинского-Владимирского: \[
f = A\delta(1 - \displaystyle\frac{\beta_x x'^2 + 2\alpha_x x x' + \gamma_x x^2}{\epsilon_x} - \displaystyle\frac{\beta_y y'^2 + 2\alpha_y y y' + \gamma_y y^2}{\epsilon_y} ),
\] где \(А\) - инвариант Куранта-Снайдера. Полуоси эллипса: \[
a = \sqrt{\epsilon_x \beta_x}, b = \sqrt{\epsilon_y \beta_y}. 
\] Поле получается линейно внутри заряженного эллиптического цилиндра:
\[
E_x = \displaystyle \frac{4I}{\upsilon}\frac{x}{a(a+b)},
\] \[
E_y = \displaystyle \frac{4I}{\upsilon}\frac{y}{b(a+b)}.
\] Проверим, что \(\nabla \vec{E} = 4\pi\rho:\) \[
\displaystyle I = \rho \upsilon \pi ab,
\] \[
\nabla \vec{E} = \displaystyle \frac{4I(a+b)}{\pi(a+b)ab} = \displaystyle \frac{4I}{\pi ab} = 4\pi\rho.
\] Так как поля линейные, они добавятся к полям фокусирующей линзы.
Подставим в уравнение огибающей
\(\displaystyle a = \sqrt \epsilon_x w_x, b = \sqrt \epsilon_y w_y:\) \[
a'' + k_{xt} a - \frac{\epsilon_x^2}{a^3} = 0,
\] где \(k_{xt} = k_x + k_{xsc}\) - полная жесткость, \(k_x\) -
жесткость линзы, а
\(\displaystyle k_{xsc} = \frac{4I}{I_a (\beta\gamma)^3}\frac{1}{a(a+b)}.\)
В итоге получаем систему уравнений, связанных через пространственный
заряд: \[
\begin{equation*}
 \begin{cases}
   \displaystyle a'' + k_xa - \frac{4I}{I_a (\beta\gamma)^3}\frac{1}{(a+b)} - \frac{\epsilon_x^2}{a^3} = 0 ,
   \\
   \displaystyle b'' + k_yb - \frac{4I}{I_a (\beta\gamma)^3}\frac{1}{(a+b)} - \frac{\epsilon_y^2}{b^3} = 0.
 \end{cases}
\end{equation*}
\]

    \hypertarget{ux43aux43eux43bux438ux447ux435ux441ux442ux432ux435ux43dux43dux44bux439-ux43aux440ux438ux442ux435ux440ux438ux439-ux43fux440ux438ux43cux435ux43dux438ux43cux43eux441ux442ux438-ux43fux440ux438ux431ux43bux438ux436ux435ux43dux438ux44f-ux43bux430ux43cux438ux43dux430ux440ux43dux43eux441ux442ux438-ux442ux435ux447ux435ux43dux438ux44f}{%
\subsubsection{Количественный критерий применимости приближения
ламинарности
течения}\label{ux43aux43eux43bux438ux447ux435ux441ux442ux432ux435ux43dux43dux44bux439-ux43aux440ux438ux442ux435ux440ux438ux439-ux43fux440ux438ux43cux435ux43dux438ux43cux43eux441ux442ux438-ux43fux440ux438ux431ux43bux438ux436ux435ux43dux438ux44f-ux43bux430ux43cux438ux43dux430ux440ux43dux43eux441ux442ux438-ux442ux435ux447ux435ux43dux438ux44f}}

    Данная система уравнений позволяет учесть 2 эффекта, мешающих
сфокусировать пучок в точку - конечность эммитанса и пространственный
заряд. Работают члены одинаково, поэтому можно сравнить эти величины.
Когда ток \(I\) малый - слабое отталкивание, если ток \(I\) большой -
сильное отталкивание, следовательно, эммитанс можно откинуть и считать
течение ламинарным. Очевидно, количественный критерий применимости
ламинарности течения выглядит так: \[
\displaystyle \sqrt{\frac{2I}{I_a(\beta\gamma)^3}} \gg \sqrt{\frac{\epsilon}{\beta}}.
\] Видно, что пространственный заряд влияет на огибающую больше там, где
\(\beta\)-функция больше, а в вблизи фокуса влиянием пространственного
заряда можно пренебречь.

    \hypertarget{ux442ux435ux43eux440ux435ux43cux430-ux431ux443ux448ux430}{%
\subsubsection{Теорема
Буша}\label{ux442ux435ux43eux440ux435ux43cux430-ux431ux443ux448ux430}}

    В качестве дополнения к выводу уравнения огибающей пучка докажем важную
вспомогательную теорему, которая называется теоремой Буша. Она связывает
угловую скорость заряженной частицы, движущейся в аксиально-симметричном
магнитном поле, с магнитным потоком, охваченным окружностью с центром на
оси и проходящим через точку, в которой расположена частица.

    Рассмотрим заряд \(q\), движущийся в магнитном поле
\(\vec B = (B_r,0,B_z)\). Приравняем \$\theta \$ -составляющую силы
Лоренца к производной момента импульса по времени, деленной на \(r\): \[
F_\theta = -q(\ddot r B_z - \dot z B_r) = \frac{d}{rdt}(\gamma m r^2 \dot \theta).
\] Поток, пронизывающий площадь, охваченную окружностью радиуса \(r\),
центр которой расположен на оси, а сама она проходит через точку, в
которой расположен заряд, записывается в виде
\(\psi = \int\limits^r_0 2\pi r B_z dr\). Когда частица перемещается на
\(\vec{dl} = (dr,dz)\), скорость изменения потока, охваченного этой
окружностью, можно найти из второго уравнения Максвелла
\(\nabla \vec{B} = 0.\) Таким образом, \[
\dot\psi = 2\pi r (-B_r \dot z + B_z \dot r).
\] После интегрирования по времени из приведенных уравнений получаем
следующее выражение: \[
\dot \theta = (-\frac{q}{2\pi \gamma m r^2})(\psi - \psi_0).
\]

    \hypertarget{ux443ux440ux430ux432ux43dux435ux43dux438ux435-ux43fux430ux440ux430ux43aux441ux438ux430ux43bux44cux43dux43eux433ux43e-ux43bux443ux447ux430}{%
\subsubsection{Уравнение параксиального
луча}\label{ux443ux440ux430ux432ux43dux435ux43dux438ux435-ux43fux430ux440ux430ux43aux441ux438ux430ux43bux44cux43dux43eux433ux43e-ux43bux443ux447ux430}}

    Здесь мы запишем уравнение параксиального луча в виде, соответсвующем
системе с аксиальнорй симметрией при принятых ранее допущениях. Чтобы
вывести уравнение параксиального луча, приравняем силу радиального
ускорения силам электрическим и магнитным со стороны внешних полей.
Нужно помнить, что величина \(\gamma = \gamma(t).\) \[
\frac{d}{dt}(\gamma m \dot r) - \gamma m r (\dot \theta)^2 = q(E_r + r\dot \theta B_z).
\] Применим теорему Буша и независимость \(B_z\) от \(r:\) \[
 -\dot \theta = \frac{q}{2\gamma m}(B_z - \frac{\psi_0}{\pi r}).
\] Исключим \(\dot \theta\) и подставим
\(\displaystyle \dot \gamma \approx \frac {\beta q E_z }{mc}:\) \[
\ddot r + \frac{\beta q E_z}{\gamma m c} \dot r + \frac{q^2 B^2_z}{4\gamma ^2 m^2} r - \frac{ q^2 \psi^2_0}{4\pi^2\gamma^2 m^2} (\frac{1}{r^3}) - \frac{q E_r}{\gamma m } = 0.
\]

    \hypertarget{ux443ux440ux430ux432ux43dux435ux43dux438ux435-ux43eux433ux438ux431ux430ux44eux449ux435ux439-ux434ux43bux44f-ux43aux440ux443ux433ux43bux43eux433ux43e-ux438-ux44dux43bux43bux438ux43fux442ux438ux447ux435ux441ux43aux43eux433ux43e-ux43fux443ux447ux43aux430}{%
\subsubsection{Уравнение огибающей для круглого и эллиптического
пучка}\label{ux443ux440ux430ux432ux43dux435ux43dux438ux435-ux43eux433ux438ux431ux430ux44eux449ux435ux439-ux434ux43bux44f-ux43aux440ux443ux433ux43bux43eux433ux43e-ux438-ux44dux43bux43bux438ux43fux442ux438ux447ux435ux441ux43aux43eux433ux43e-ux43fux443ux447ux43aux430}}

    Учитывая, что: \[
\dot r = \beta c r',
\] \[
\ddot r = r'' (\dot z)^2 + r'\ddot z \approx r'' \beta^2 c^2 + r' \beta' \beta c^2.
\] А также, если в области пучка нет никаких зарядов, то, разлагая в ряд
Тейлора в окрестности оси и оставляя только первый член, с учетом
\[\nabla \vec{E} = 0\] получаем \[
E_r = -0.5 r E'_z \approx - 0.5 r \gamma'' mc^2/q.
\] Тогда окончательно можем записать уравнение огибающей для круглого
пучка радиуса \(r\) с распределением Капчинского-Владимирского с внешней
фокусировкой линейными полями: \[
\displaystyle r'' + \frac{1}{\beta^2\gamma} \gamma' r' + \frac{1}{2\beta^2\gamma}\gamma''r + kr - \frac{2I}{I_a (\beta\gamma)^3}\frac{1}{r} - \frac{\epsilon^2}{r^3} = 0 ;
\] и для эллиптического пучка: \[
\begin{equation*}
 \begin{cases}
   \displaystyle a'' + \frac{1}{\beta^2\gamma} \gamma' a' + \frac{1}{2\beta^2\gamma}\gamma''a + k_xa - \frac{4I}{I_a (\beta\gamma)^3}\frac{1}{(a+b)} - \frac{\epsilon_x^2}{a^3} = 0 ,
   \\
   \displaystyle b'' + \frac{1}{\beta^2\gamma} \gamma' b' + \frac{1}{2\beta^2\gamma}\gamma''b + k_yb - \frac{4I}{I_a (\beta\gamma)^3}\frac{1}{(a+b)} - \frac{\epsilon_y^2}{b^3} = 0.
 \end{cases}
\end{equation*}
\]

    \hypertarget{ux43cux430ux433ux43dux438ux442ux43dux44bux435-ux43bux438ux43dux437ux44b}{%
\subsubsection{Магнитные
линзы}\label{ux43cux430ux433ux43dux438ux442ux43dux44bux435-ux43bux438ux43dux437ux44b}}

Фокусирующими элементами могут являться: соленоиды и магнитные
квадрупольные линзы.

\hypertarget{ux441ux43eux43bux435ux43dux43eux438ux434ux44b}{%
\subsubsection{Соленоиды}\label{ux441ux43eux43bux435ux43dux43eux438ux434ux44b}}

\(k_x = k_y = k_s\) - жесткость соленоида: \[
k_s = \left ( \frac{eB_z}{2m_ec\beta\gamma} \right )^2 = \left ( \frac{e B_z}{2\beta\gamma\cdot 0.511\cdot 10^6 e \cdot \mathrm{volt}/c} \right )^2 =
\left ( \frac{cB_z[\mathrm{T}]}{2\beta\gamma\cdot 0.511\cdot 10^6 \cdot \mathrm{volt}} \right )^2.
\]

\hypertarget{ux43aux432ux430ux434ux440ux443ux43fux43eux43bux438}{%
\subsubsection{Квадруполи}\label{ux43aux432ux430ux434ux440ux443ux43fux43eux43bux438}}

\(k_q = \displaystyle\frac{eG}{pc}\) - жесткость квадруполя, где
\(G = \displaystyle\frac{\partial B_x}{\partial y} = \displaystyle\frac{\partial B_y}{\partial x}\)
- градиент магнитного поля, причем \(k_x = k_q, k_y = -k_q.\) \[
k_q = \left ( \frac{eG}{m_ec\beta\gamma} \right ) = \left ( \frac{eG}{\beta\gamma\cdot 0.511\cdot 10^6 e \cdot \mathrm{volt}/c} \right ) =
\left ( \frac{cG}{\beta\gamma\cdot 0.511\cdot 10^6 \cdot \mathrm{volt}} \right ).
\]

    \hypertarget{ux43fux440ux43eux434ux43eux43bux44cux43dux430ux44f-ux434ux438ux43dux430ux43cux438ux43aux430-ux43fux443ux447ux43aux430}{%
\subsubsection{Продольная динамика
пучка}\label{ux43fux440ux43eux434ux43eux43bux44cux43dux430ux44f-ux434ux438ux43dux430ux43cux438ux43aux430-ux43fux443ux447ux43aux430}}

    Уравнение на продольную динамику пучка можно решить независимо от
уравнения на огибающую, чтобы в уравнении на огибающую уже использовать
готовую функцию энергии пучка от \(z\). Считая, что скорость электрона
достаточно близка к скорости света и следовательно его продольная
координата \(z \approx ct\), а импульс \(p_z \approx \gamma mc\)

    \[
\frac{d\gamma}{dz} \approx \frac{eE_z}{mc^2},
\]

    Тогда достаточно один раз проинтегрировать функцию \(E_z(z)\):

    \hypertarget{ux440ux435ux448ux435ux43dux438ux435-ux443ux440ux430ux432ux43dux435ux43dux438ux44f-ux43eux433ux438ux431ux430ux44eux449ux435ux439-ux434ux43bux44f-ux44dux43bux43bux438ux43fux442ux438ux447ux435ux441ux43aux43eux433ux43e-ux43fux443ux447ux43aux430-ux441-ux444ux43eux43aux443ux441ux438ux440ux443ux44eux449ux438ux43cux438-ux44dux43bux435ux43cux435ux43dux442ux430ux43cux438}{%
\subsection{Решение уравнения огибающей для эллиптического пучка с
фокусирующими
элементами}\label{ux440ux435ux448ux435ux43dux438ux435-ux443ux440ux430ux432ux43dux435ux43dux438ux44f-ux43eux433ux438ux431ux430ux44eux449ux435ux439-ux434ux43bux44f-ux44dux43bux43bux438ux43fux442ux438ux447ux435ux441ux43aux43eux433ux43e-ux43fux443ux447ux43aux430-ux441-ux444ux43eux43aux443ux441ux438ux440ux443ux44eux449ux438ux43cux438-ux44dux43bux435ux43cux435ux43dux442ux430ux43cux438}}

Уравнение огибающей для эллиптического пучка с полуосями \(a, b\) с
распределением Капчинского-Владимирского с внешней фокусировкой
линейными полями: \[
\begin{equation*}
 \begin{cases}
   \displaystyle a'' + \frac{1}{\beta^2\gamma} \gamma' a' + \frac{1}{2\beta^2\gamma}\gamma''a + k_qa - \frac{2P}{(a+b)} - \frac{\epsilon_x^2}{a^3} = 0 ,
   \\
   \displaystyle b'' + \frac{1}{\beta^2\gamma} \gamma' b' + \frac{1}{2\beta^2\gamma}\gamma''b - k_qb - \frac{2P}{(a+b)} - \frac{\epsilon_y^2}{b^3} = 0.
 \end{cases}
\end{equation*}
\]

    Пусть \$\displaystyle x = \frac{da}{dz}, y = \frac{db}{dz},
\displaystyle \frac{d\gamma }{dz}\approx \frac{e E_z}{m c^2}, k =
\frac{1}{R} - \$ кривизна, тогда \[
\displaystyle
\left\{\begin{matrix}
\displaystyle \frac{dx}{dz}= -  \frac{1}{\beta^2\gamma} \gamma' a' - \frac{1}{2\beta^2\gamma}\gamma''a - k_qa + \frac{2P}{(a+b)} + \frac{\epsilon_x^2}{a^3}\\ 
 \displaystyle\frac{da}{dz} =  x\\
 \displaystyle \frac{dy}{dz}= -  \frac{1}{\beta^2\gamma} \gamma' b' - \frac{1}{2\beta^2\gamma}\gamma''b + k_qb + \frac{2P}{(a+b)} + \frac{\epsilon_x^2}{b^3}\\ 
 \displaystyle\frac{db}{dz} =  y
\end{matrix}\right.
\] Пусть \$\vec X =

\begin{bmatrix}
x \\
a \\
y \\
b
\end{bmatrix}

\$, теперь составим дифференциальное уравнение \(X' = F(X).\)

    \hypertarget{ux43bux438ux442ux435ux440ux430ux442ux443ux440ux430}{%
\subsubsection{Литература}\label{ux43bux438ux442ux435ux440ux430ux442ux443ux440ux430}}

    \begin{enumerate}
\def\labelenumi{\arabic{enumi}.}
\tightlist
\item
  Дж. Лоусон ``Физика пучков заряженных частиц''
\item
  Н.А. Винокуров ``Лекции по электронной оптике для ускорительных
  физиков''
\end{enumerate}


    % Add a bibliography block to the postdoc
    
    
    
    \end{document}
